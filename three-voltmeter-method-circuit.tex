%%%%%%%%%%%%%%%%%%%%%%%%%%%%%%%%%%%%%%%%%%%%%%%%%%%%%%%%%%%%%%%
%
% Circuit Diagram for Three Voltmeter Method
%
%%%%%%%%%%%%%%%%%%%%%%%%%%%%%%%%%%%%%%%%%%%%%%%%%%%%%%%%%%%%%%%
% Author: euikook <euikook@gmail.com>  (October 2023)
\documentclass[border=3pt,tikzcircuit]{standalone}


\usepackage{physics}
\usepackage[outline]{contour} % glow around text

\usepackage{tikz}
  \usetikzlibrary{angles,quotes} % for pic

\usepackage{circuitikz}

\begin{document}
    \begin{circuitikz}[scale=1.4]
      \draw
      (2, 0) -- (0,0) to[sinusoidal voltage source, l=$V_{in}$] (0,4) -- (2, 4)
      %to[R=R, -*] (4,4)
      (2, 0) to[rmeterwa, t=V, l=$V_{1}$, *-*] (2,4)
      (2, 4) to [R, l=$R$] (4, 4) 
      to[rmeterwa, t=V, l=$V_{3}$, *-*] (4,0) -- (2, 0)
      (4, 4) -- (6, 4) to [generic, l=$Z_{L}$] (6, 0) -- (4, 0)
      (2, 4) -- (2, 5.5) to[rmeterwa, t=V, l=$V_{2}$] (4, 5.5) -- (4, 4)
      %(4,4) -* (6,4)
      %to[ammeter, l=$I_{mis}$] (6,2) to[R=$R_x$, -*] (6,0) -- (4,0)
      ;
    \end{circuitikz}
\end{document}
