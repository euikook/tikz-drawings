%%%%%%%%%%%%%%%%%%%%%%%%%%%%%%%%%%%%%%%%%%%%%%%%%%%%%%%%%%%%%%%
%
% Circuit Diagram for Three Voltmeter Method
%
%%%%%%%%%%%%%%%%%%%%%%%%%%%%%%%%%%%%%%%%%%%%%%%%%%%%%%%%%%%%%%%
% Author: euikook <euikook@gmail.com>  (October 2023)
\documentclass[border=3pt,tikzcircuit]{standalone}


\usepackage{physics}
\usepackage[outline]{contour} % glow around text

\usepackage{tikz}
  \usetikzlibrary{angles,quotes} % for pic

\usepackage{circuitikz}

\begin{document}
    \begin{circuitikz}[scale=1.4]
      \draw
      (0,0) to[sinusoidal voltage source, l=$V_{in}$, o-o] (0,4)
      to[rmeterwa, t=A, l=$I_{1}$] (3, 4)
      to[rmeterwa, t=A, l=$I_{2}$, *-] (3, 1.5)
      (3, 2.2) to [R, l=$R$, -*] (3, 0) -- (0, 0)       
      (3, 4) to[rmeterwa, t=A, l=$I_{3}$] (6,4)
      to [generic, l=$Z_{L}$] (6, 0) -- (0,0)

      (2, 4) to [short, i=$I_1$] (3, 4)
      (3, 4) to [short, i=$I_2$] (3, 3)
      (3, 4) to [short, i=$I_3$] (4, 4)

      

      % to [short, i=$I_1$] (1, 4) % stator current
      % (2, 4) to [short, i=$I_3$, *-] (3, 4)
      % (2, 0) to[rmeterwa, t=V, l=$V_{1}$, *-*] (2,4)
      % (2, 4) to [R, l=$R$] (4, 4) 
      % to[rmeterwa, t=V, l=$V_{3}$, *-*] (4,0) -- (2, 0)
      % (4, 4) -- (6, 4) to [generic, l=$Z_{L}$] (6, 0) -- (4, 0)
      % (2, 4) -- (2, 5.5) to[rmeterwa, t=V, l=$V_{2}$] (4, 5.5) -- (4, 4)
      % %(4,4) -* (6,4)
      % %to[ammeter, l=$I_{mis}$] (6,2) to[R=$R_x$, -*] (6,0) -- (4,0)
      ;
    \end{circuitikz}
\end{document}
