%%%%%%%%%%%%%%%%%%%%%%%%%%%%%%%%%%%%%%%%%%%%%%%%%%%%%%%%%%%%%%%
%
% Circuit Diagram for Two Wattmeter Method
%
%%%%%%%%%%%%%%%%%%%%%%%%%%%%%%%%%%%%%%%%%%%%%%%%%%%%%%%%%%%%%%%
% Author: euikook <euikook@gmail.com>  (October 2023)
\documentclass[border=3pt,tikzcircuit]{standalone}


\usepackage{physics}
\usepackage[outline]{contour} % glow around text

\usepackage{tikz}
  \usetikzlibrary{angles,quotes} % for pic

\usepackage{circuitikz}

\begin{document}
    \begin{circuitikz}[scale=1.4]
      \draw
      (-1, 5) to[short, o-] (0, 5) to[cute inductor] (4, 5) 
      (1, 5) to[short, *-] (1, 4.5) to[R] (3, 4.5) to[short, -*] (3, 2)
      
      (-1, 2) to[short, o-] (4, 2) -- (4.268, 2)

      (1, 0) to[short, *-] (1, 0.5) to[R] (3, 0.5) to[short, -*] (3, 2)
      (-1, 0) to[short, o-] (0, 0) to[cute inductor](4, 0) to [short] (7.732, 0) -- (7.732, 2)

      (4, 5) -- (6, 5) to [generic, *-*, l=$Z_a$] (6, 3)
      to [generic, *-*, l=$Z_b$] (4.268, 2)
      (6, 3) to[generic, *-*, l=$Z_c$] (7.732, 2)


      (-1, 5) node[anchor=east] {A}
      (-1, 2) node[anchor=east] {B}
      (-1, 0) node[anchor=east] {C}

      (3, 5) to [short, i=$I_a$] (4, 5)
      (3, 2) to [short, i=$I_b$] (4, 2)
      (3, 0) to [short, i=$I_c$] (4, 0)

      % Wattmeter Circle
      (2, 1.3) node[anchor=north] {$W_2$}
      (2, 0.25) circle [radius = 20pt] (2, 0.25)

      (2, 5.8) node[anchor=north] {$W_1$}
      (2, 4.75) circle [radius = 20pt] (2, 4.75)
      ;
    \end{circuitikz}
\end{document}
